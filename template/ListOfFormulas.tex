%Root document for merging
% !TEX root = MainDoc.tex
 % Explanation:
 %\acro{KÜRZEL}[ABKÜRZUNG]{\acrounit{SI-EINHEIT}BESCHREIBUNG}

\addchap{Formelzeichenverzeichnis}

\textbf{Lateinische Symbole} \vspace{0.5cm}
  
\begin{tabular}{p{2cm} p{2cm} l}
$A$ 						& \si{\milli\meter^2}							&Fläche \\

$B$						& \si{\milli\meter}							&Außenbreite\\

$b$						& \si{\milli\meter}							&Innenbreite \\

$b_{\rm{SP}}$		& \si{\milli\meter}							&Schneidplattenbreite \\

$c$						& \si{\newton\per\micro\meter}			&Steifigkeit \\

$c_{\rm{D}}$			& \si{\newton\per\micro\meter}	 		&Drucksteifigkeit \\

$c_{\rm{B}}$			& \si{\newton\per\micro\meter}	 		&Biegesteifigkeit \\

$c_{\rm{T}}$			& \si{\newton\meter\per\circ}	 		&Torsionssteifigkeit \\

$c_{\rm{D_{\rm{indirekt}}}}$			& \si{\newton\per\circ}	 		&Indirekte Drucksteifigkeit \\

$c_{\rm{T_{\rm{indirekt}}}}$			& \si{\newton\meter\per\micro\meter}	 		&Indirekte Torsionssteifigkeit \\

$D$						& \si{\milli\meter}							&Außendurchmesser \\

$D_{\rm{KMK}}$			& \si{\milli\meter}							&Außendurchmesser Kühlmittelkanal \\

$D_{\rm{S}}$			& \si{\milli\meter}							&Außendurchmesser Schaft \\

$D_{\rm{SK}}$		& \si{\milli\meter}							&Außendurchmesser Schneidkopf \\
                    
$d$						&	\si{\milli\meter}							&Innendurchmesser \\

$d_{\rm{S}}$ 			& \si{\milli\meter}							&Innendurchmesser Schaft \\

$E$						& \si{\newton\per\milli\meter^2}		&E-Modul \\

$F$						& \si{\newton}								&Kraft \\

$F_{\rm{D}}$			& \si{\newton}								&Druckkraft \\

$F_{\rm{B}}$			& \si{\newton}								&Biegekraft \\

$G$						& \si{\newton\per\milli\meter^2}		&Schubmodul \\

$I$						& \si{\milli\meter^4}							&Axiales Flächenmoment \\

$I_{\rm{P}}	$			& \si{\milli\meter^4}							&Polares Flächenmoment \\

$l$						& \si{\milli\meter}								&Länge \\

$l_{\rm{S}}$			& \si{\milli\meter}							&Länge Schaft \\

$M_{\rm{C}}$			& \si{\newton\meter}						&Schnittmoment \\

$M_{\rm{T}}$			& \si{\newton\meter}						&Torsionsmoment \\

$r$						& \si{\milli\meter}							&Radius \\

\end{tabular} \vspace{0.5cm}

\textbf{Griechische Symbole} \vspace{0.5cm}

\begin{tabular}{p{2cm} p{2cm} l}
$\delta$				& \si{\micro\meter\per\newton}			&Nachgiebigkeit \\

$\epsilon$				& -												&Dehnung \\

$\nu$					& - 												&Querkontraktionszahl \\

$\sigma$				& \si{\newton\per\milli\meter^2}		&Spannung \\

$\varphi$				& \si{\circ}										&Verdrehwinkel \\


\end{tabular}